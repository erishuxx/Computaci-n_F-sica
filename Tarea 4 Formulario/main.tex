\documentclass[letterpaper,12pt]{article}
%____________________________PAQUETERIAS_________________________
\usepackage[utf8]{inputenc}
\usepackage{amsmath}
\usepackage{amssymb}
\usepackage{latexsym}
\usepackage{mathrsfs}
\usepackage[dvipsnames]{xcolor}
\usepackage{dsfont}
\usepackage{fancyhdr}
    \pagestyle{fancy}
    
           \fancyhf{}
\cfoot{\thepage}
\lfoot{Erick Yahel Pérez Álvarez}


%_________________________INICIO DEL DOCUMENTO_____________________________________



\begin{document}

{\centering\huge{{Ecuaciones Primordiales}\\


Pérez Álvarez Erick Yahel\\
27/Oct/2022}\\}


\newpage
\lhead{Física}
\section{Física}
\begin{itemize}

    \item [\star]Periodo (T): tiempo que tarda el péndulo en realizar una oscilación completa. Frecuencia (v): número de oscilaciones que realiza el péndulo en un segundo. Esto quiere decir que la amplitud del movimiento se irá reduciendo progresivamente, pero el periodo seguirá siendo el mismo.
    
                                     $$ 	T=2\pi\sqrt{\dfrac{l}{g }}$$
    
    
    \item [\bigstar]La velocidad de escape es la velocidad inicial que hay que imprimirle a un objeto cualquiera para alejarse indefinidamente de un cuerpo o sistema más masivo al cual le vincula únicamente la gravedad. La velocidad de escape (ve) depende de la masa (M) del cuerpo o sistema masivo y de la distancia que separa los centros de masas de ambos (r) a través de la siguiente ecuación donde G es la constante de gravitación universal
    
                                     $$v_e=\dst\sqrt{\frac{2GM}{r}}=\sqrt{2gr}$$
    
    
    \item [\bullet]En fluidodinámica la velocidad límite o velocidad final es la velocidad máxima que alcanza un cuerpo moviéndose en el seno de un fluido infinito bajo la acción de una fuerza constante. Un ejemplo es el caso de la velocidad límite alcanzada por un paracaidista en caída libre que cae desde suficiente altura.
                      
                      
                      $$V_l=\dfrac{2g(\rho_e-\rho_m)R^2}{9\eta}$$
    
    
    
    
    \item [\circ]La segunda ley de Newton define la relación exacta entre fuerza y aceleración matemáticamente. La aceleración de un objeto es directamente proporcional a la suma de todas las fuerzas que actúan sobre él e inversamente propocional a la masa del objeto, Masa es la cantidad de materia que el objeto tiene.
    
                    $$\Vec{F}=m\Vec{a}$$
    
    \item [\triangleright]Para una esfera de radio R moviéndose en un flujo no turbulento dentro de un fluido de viscosidad η, la velocidad límite viene dada por la ley de Stokes, que postula que la fuerza de resistencia Fr es proporcional a la velocidad. En ese caso la velocidad límite viene dada por: 
    
                  $$v_{\infty}=\frac{F_{r}}{6\pi\eta R}$$
                  
                  
    \item[\lhd]La energía cinética de una masa puntual depende de su masa m m y sus componentes del movimiento. Se expresa en julios o joules (J). 1 J = 1 kg·m²/s2. Estos son descritos por la velocidad v v de la masa puntual
    
                  $$E_c=\frac12mv^2$$
    
    
    \item[\square]La potencia mecánica es la cantidad de fuerza aplicada a un cuerpo en relación a la velocidad con que se aplica.
     
                  $$P=\vec{F}\cdot \vec v = \dfrac{E_m}{t}$$
    
    
    \item[\hbar]La potencia rotacional instantánea se define como el producto del momento actuante por la velocidad angular 
     
                  $$P=\vec{M}\cdot \vec {\omega}$$
    
     \item[\square]Newton dedujo que la fuerza con que se atraen dos cuerpos tenía que ser proporcional al producto de sus masas dividido por la distancia entre ellos al cuadrado. 
        $$F=G \frac{m_{1}m_{2}}{r^{2}}$$

    \item[\lhd]La magnitud de cada una de las fuerzas eléctricas con las que interactúan dos cargas puntuales en reposo es directamente proporcional al producto de la magnitud de ambas cargas e inversamente proporcional al cuadrado de la distancia que las separa y tiene la dirección de la línea que las une. La fuerza es de repulsión si las cargas son de igual signo, y de atracción si son de signo contrario.
    
    
    $$\vec{F}=K \frac{q_{1} q_{2}}{r^{2}}$$





\end{itemize}



\newpage

\lhead{Geometria Analitica}

\section{Geometria Analitica}

\begin{itemize}
  
 \item [\star]La pendiente de una recta es la tangente del ángulo  que forma la recta con la dirección positiva del eje de abscisas.
 
            $$ 	m=\frac{y_2-y_1}{x_2-x_1} $$
    
    \item [\bigstar]El vértice de la parábola es el punto
    
            $$ 	f(x) = a(x - h)^2 + k $$
    
    
    \item [\bullet]Dado un vector de un espacio vectorial euclídeo, la norma de un vector es definida como la distancia (en línea recta) entre dos puntos A y B que delimitan al vector.
    
            $$||\vec{u}||=\sqrt{a^{2}+b^{2}}$$
    
    
    
    
    \item [\circ] La fórmula de la distancia sirve para calcular la distancia entre cualesquiera dos puntos.
    
             $$ d=\sqrt{(x_1 - x_2)^2 + (y_1 - y_2)^2} $$
    
    \item [\triangleright]La ecuación general de la recta describe el comportamiento de todas las rectas existentes en el plano cartesiano.
    
            $$Ax+By+C=0$$
                  
                  
    \item[\lhd]El punto medio entre dos puntos es un punto que tiene coordenadas que se ubican exactamente en la mitad de los dos puntos.
      
            $$Pm=\left( \frac{x_{1}+x_{2}}{2},\frac{y_{1}+y_{2}}{2}\right)$$
    
    
    \item[\square]La longitud de la hipotenusa es igual a la raíz cuadrada de la suma del área de los cuadrados de las respectivas longitudes de los catetos. 
        
            $$c^{2}=a^{2}+b^{2}$$
\end{itemize}

\newpage

\lhead{Algebra}

\section{Algebra}

\begin{itemize}
    \item[\hbar]La fórmula cuadrática nos ayuda a resolver cualquier ecuación cuadrática. 
    $$a x^2 + b x + c = 0 $$
    \item[\lhd]La fórmula cuadrática nos ayuda a resolver cualquier ecuación cuadrática.
    $$x = \frac {-b \pm \sqrt {b^2 - 4ac}}{2a} $$
    \item [\triangleright]Una ecuación cúbica es una ecuación algebraica de tercer grado. 
    $$ 	a x^3 + b x^2 + c x + d = 0 $$
    \item [\circ]Todo númer complejo puede expresarse de la forma binómica
     $$z_{1}=a+ib$$
\end{itemize}
\end{document}
